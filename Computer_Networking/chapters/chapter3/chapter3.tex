\chapter{Transport Layer}

\section{Introduction and Transport-Layer Services}

\hf A transport-layer protocl provides for \tbi{logical communication} between application processes running
on different hosts. Transport-layer protocols are implemented in the end systems but not in network
routers.\\

On sending side, the transport layer converts the \tbi{message} it receives into \tbi{segments}.
This is done by breaking the \tbi{message} into smaller chunks and adding a transport-layer header to each
chunk to create the \tbi{segment}. Then the transport layer passes the segment to the network-layer at the sending
end system, and sent to the destination. It is important to note that network routers act only
on the network-layer fields of the datagram; that is, they do not examine the fields of the transport-layer
segment encapsulated with the datagram.\\

On receiving side, the network-layer extracts the transport-layer segment from the datagram and passes the 
segment up to the transport layer. The transport layer then processes the received segment, making
the data in the segment available to the receiving application.

\subsection{Relationship Between Transport and Network Layers}

Transport-layer protocol provides logical communication between processes running on different hosts.\\

Network-layer protocol provides logical communication between hosts.\\

The possible services can be provided by a transport-layer protocol is constrained by the possible services
that the network-layer protocol --- the Internet Protocol. If the network-layer protocol cannot provide
delay or bandwidth guarantees for transport-layer segments send between hosts, then the transport-layer protocol cannot provide
delay or bandwidth guarantees for application messages sent between processes.\\

Nevertheless, certain services can be offered by a transport protocol even when
the underlying network protocol doesn’t offer the corresponding service at the net-
work layer. For example, a transport protocol can offer reliable data transfer service even IP cannot.
A transport protocol can use encryption even IP cannot.


\subsection{Overiew of the Transport Layer in the Internet}

\hf The Internet, and more generally TCP/IP network, makes two distinct transport-layer protocols
available to the application layer.\\

One of these protocol is \tbi{UDP}(User Datagram Protocol), which provides an unreliable, connectionless
service to the invoking application.\\

The second of these protocol is \tbi{TCP}(Transmission Control Protocol), which provides a reliable, connection-oriented service to the invoking
application.\\

The Internet's network-layer protocol is called IP(Internet Protocol), which provideslogical communication between hosts.
The IP provides "best-effort delivery service".\\

The most fundamental responsibility of UDP
and TCP is to extend IP’s delivery service between two end systems to a delivery
service between two processes running on the end systems.\\

Minimal transport-layer services:
\begin{enumerate}
    \item Process-to-process data delivery:Extending host-to-host
    delivery(IP) to process-to-process delivery by transport-layer \tbi{multiplexing} and
    \tbi{demultiplexing}.
    \item  Error checking: UDP and TCP also provide integrity checking by including \tbi{error-detection fields} in their segments’ headers.
\end{enumerate}



\section{Multiplexing and Demultiplexing}

\hf Demultiplexing: Deliverying the data in a transport-layer segment to the correct socket.\\

Multiplexing: Gathering data chunks at the source host from different sockets, encapsulating each
data chunk with header information to create segments, and passing the segments to the network layer.\\

\insertImage{width=0.8\textwidth}{chapters/chapter3/transport_layer_multiplexing_demultiplexing.png}{Transport-layer multiplexing and demultiplexing}{c3_transport_multiplexing}


\subsection{How to achieve multiplexing and demultiplexing}

\hf Transport-layer multiplexing/demultiplexing requires
\begin{enumerate}
    \item Source port: Unique Identifiers for sockets
    \item Destination port: Each segment have special fields that indicate socket to which the segment is to be delivered.
\end{enumerate}

\insertImage{width=0.4\textwidth}{chapters/chapter3/Source_destination_in_segment.png}{Source and destination port-number fields in a transport-layer
segment}{c3_source_destination_port}


Each port number is a 16-bit number, ranging from 0 to 65535. The port numbers ranging \tbi{from 0 to 1023}
are called \tbi{well-know port numbers} and \tbi{are reserved} for certain applications.\\
\newline


\subsubsection{Connectionless(UDP) Multiplexing and Demultiplexing}

\hf UDP is fully identified by two-tuple consisting	of 
\begin{enumerate}
	\item Destination IP address
	\item Destination port number
\end{enumerate}

\insertImage{width=0.6\textwidth}{chapters/chapter3/UDP_multiplexing.png}{Connectionless Multiplexing and Demultiplexing}{c3_connectionless_multi_demulti}

UDP server processes all clients via the same socket.

\subsubsection{Connection-Oriented Multiplexing and Demultiplexing}

\hf TCP is identified by four-tuple consisting of
\begin{enumerate}
	\item Source IP address
	\item Source port number
	\item Destination IP address
	\item Destination port number
\end{enumerate}

\insertImage{width=0.6\textwidth}{chapters/chapter3/TCP_demultiplexing.png}{Connection-Oriented Multiplexing and Demultiplexing}{c3_connection_oriented_multi_demulti}

TCP maintains a seperate socket for every connection. Why?\\

Because TCP implements reliability, it
is easier with separate sockets. For example, when a message arrives
at a socket, it is easy to simply see the source IP/port associated
with that socket and send ACK. If you want to send a reply to a UDP
message, you need to extract the source IP and port from the UDP
header for every message.\\


\section{Connectionless Transport: UDP}

\hf Basically, UDP pass application messages directly to the network layer after providing a multiplexing/demultiplexing service and some light error checking. UDP does just about as little as a transport protocol can do.\\

\utbi{Why would one choose UDP rather than TCP?}\\

\begin{enumerate}
	\item Real-time applications cannot tolerate delay but can endure data loss.
	\item Non-connection establishment: TCP connection-establishment delay is heavy.
	\item No connection state: Can support many more active clients when the application runs over UDP rather than TCP.
	\item Small packet header overhead: TCP only has 20 bytes of overhead. In comparison with TCP, UDP has much less header overhead --- 8 bytes of overhead.
\end{enumerate}

\utbi{Why UDP is not ideal sometimes?}

\begin{enumerate}
	\item UDP has no congestion control: lead to a heavy congested network.
\end{enumerate}

It is also possible for an application to have reliable data transfer when using UDP by building reliability into the application itself. But this is nontrivial task that would keep developers busy.  \underline{Just use TCP if you need reliability}.



\subsection{UDP Segment Structure}

\insertImage{width=0.45\textwidth}{chapters/chapter3/UDP_structure.png}{UDP Segment Structure}{c3_udp_segment_structure}
