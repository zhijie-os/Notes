\chapter{Math}

\section{label equation and refer to it}

\begin{equation}
    x+y=z
    \label{math_addition}
\end{equation}

% alignment of math equations
\begin{align}
    \label{math_align}
    f(x) &= a+b\\
         &= c   
\end{align}

% array only has one label
% array can only be used in mathematic mode
\begin{equation} % ccc means three cols and everything in the center
    \begin{array}{ccc}
        a   &b   &c \\ % & to indicate a token
        d
    \end{array}
\end{equation}

% a matrix is like an array with parenthesis.
\begin{equation}
    \left(
        \begin{array}{cccc}
            1 & 2 & 3\\
            4 & 5 & 6\\
            7 & 8 & 9
        \end{array}
    \right)
\end{equation}

% alternative matrix creation
% pmatrix <=> parenthesis matrix
% bmatrix <=> bracket matrix
\begin{equation}
    \label{math_parenthesis_matrx}
    \begin{pmatrix}
        1 & 2\\
        3 & 4
    \end{pmatrix}
\end{equation}

% number of columns is auto detected, default centered
\begin{equation}
    \begin{bmatrix}
        1 & 2\\
        3 & 4
    \end{bmatrix}
\end{equation}\label{math_bracket_matrix}

% use different math fonts: package amssymb is needed 
\begin{equation}
    \begin{array}{ccc}
        \mathbb{R} &\mathbb{Z} &\mathbb{N}\\
        \mathcal{R} &\mathcal{Z} &\mathcal{N}
    \end{array}
\end{equation}

% some math symbol
\begin{equation}
    \sum; \int; \iiint; \nabla; \cdot; \times; \otimes
\end{equation}


The first equation is \eqref{math_addition}.


The align looks like \eqref{math_align}.


The parenthesis matrix: \eqref{math_parenthesis_matrx}

The bracket matrix: \eqref{math_bracket_matrix}

